\chapter{Knowledge base completion}

Source: \cite{kg-refinement-survey}

\section{Knowledge graphs}

Freebase: closed. Wikidata: data migrated from Freebase.
DBpedia: automatically extracted from Wikipedia.
YAGO, NELL, ...
Knowledge Vault: uses information extraction from unstructured data.

\section{Methods}

Completion: adding new facts.
Error detection: identifying wrong information in graph.

TODO: What about non-edges? (e.g. population = 10M, ...)

Knowledge base completion is the process of automatically extending an
incomplete knowledge base by adding missing entities and relations.
Methods for knowledge base completion are \em{internal} and \em{external}.

Internal methods only use preexisting features of the graph to predict unseen
but true edges.

External methods additionally use a corpus of text, from which they extract
relations not necessarily supported by the prior knowledge base.

Knowledge bases: tabulka vypadá divně. Proč je KV menší než KG?
Jak je velká DBpedie, Wikidata? Proč jsou Wikidata menší než Freebase?
Jak dělali migraci na Freebase?

\subsection{Completion - Internal methods}

Predicting type of entity: classification (supervised).
Predicting existence of relations between entities (e.g. born in city in Germany
=> German).
Statistical methods (association rule mining). Type prediction, used by DBpedia.

\subsection{Completion - External methods}

Classification. (e.g. k-NN on Wiki links)

NLP methods. Predicting types of entities by classification.
Distant supervision.

[50], [4]: extending Freebase, DBpedia with Wikipedia as external data

Some methods with semi-structured data.

\section{Evaluation}

Knowledge graph as silver standard (replicating missing edges).
Measuring recall, precision, F-measure.
Can use other KG as partial gold standard.
Or ex-post evaluations by humans.
